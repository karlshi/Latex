\documentclass[a4paper, 11pt]{article}

\usepackage{amssymb}
\usepackage{graphicx}
\usepackage{setspace}
\usepackage{xcolor}


\topmargin -10mm
\oddsidemargin 0mm
\evensidemargin 0mm
\textwidth 158mm
\textheight 226mm
\parskip 0mm

\def\Response{\noindent \textbf{Response:~}}
\newcommand{\Question}[1]{\textcolor[rgb]{0.51,0.00,0.00}{#1}}
\newcommand{\PaperText}[1]{\emph{#1}}

\title{Revision of Paper TRETS-2013-0065 Imprecise Datapath Design: An Overclocking Approach}
\author{Kan Shi, David Boland and George A. Constantinides}
\date{}

\pagenumbering{arabic}

\begin{document}
\maketitle

\section*{Reviewer 3}
\begin{enumerate}
  \item \Question{The first sentence of the abstract is completely devoid of context, and may be the worst sentence in the entire article, given its importance. A much more appropriate sentence is the start of Section 1, Paragraph 4.}
            
      \Response We have modified the first two sentences of the abstract, which now read as follows:\\
           
      \PaperText{In this paper, we describe an alternative circuit design methodology when considering trade-offs between accuracy, performance and silicon area. We compare two different approaches that could trade accuracy for performance.}\\

  \item \Question{What are the chances that an entire design with all its complexities could be accelerated in this fashion? (i.e. Could the overall clock frequency be increased?)}
            
      \Response to be added.\\
      
  \item \Question{In Figure 1, you use $\mu$, whereas in the text and in Equation 4, you use $\mu_c$.  The converse is true for Section 7.4 and equations (35), (36), and (37).}
      
      \Response Thanks. It should be $\mu_c$ in Fig.1 and Equations (35), (36) and (37). It has been fixed.\\
      
  \item \Question{In Section 3.2, it would be helpful if you could illustrate or define your [-1,1) fixed-point format. Do you number your bits from k-1 to 0 or from 0 to k-1?}
      
      \Response We meant that all numbers are normalized to fractional numbers within the range [-1,1). The truncation process is illustrated in the figure below, which is not included in this manuscript due to the space limit. As seen in this figure, the original number embodies k fractional bits, while the truncated number embodies n fractional bits. Given that all bits are mutually independent and uniformly distributed, Equation (2) gives the mean value of the truncated bits. We have modified the second paragraph of Section 3.2:\\
      
      \PaperText{to be added.}\\
      
  \item \Question{In Section 3.3, for $A_i\neq B_i$ (Carry Propagation), you seem to be saying that $S_i = C_{i-1} = 0$.  If $A_i \neq B_i$ then $A_i + B_i = 1$, but that tells us nothing unless we know $C_{i-1}$, and it is not necessarily 0 as you claim.}
      
      \Response Thanks for pointing this out. In this case it should be $C_{i-1} =1$, because carry is propagated from bit $i-1$ to bit $i$. Hence $S_i=A_i+B_i+C_{i-1}=0$.
      
      This statement is modified as follows:\\
      
      \PaperText{Carry propagation: $A_i\neq B_i$, the carry propagates for this carry at bit $i$, and $C_{i-1}=1$, $S_i=0$.}\\
      
  \item \Question{For $A_i = B_i$ (either 0 or 1), there is no carry annihilation if their value is 1, and $S_i = C_{i-1} = 1$ seems to be a condition, not a consequence.}
      
      \Response This statement is only applied to determine the annihilation of an existing carry chain, i.e. $C_{i-1}=1$. In this case if $A_i=B_i=1$, the current carry chain is annihilated, meanwhile a new carry chain is generated. If $A_i=B_i=0$, the current carry chain is annihilated but without generating a new chain. In both situations we have $S_i=C_{i-1}=1$.

  \item \Question{In 3.2, we have $k$ as the full RCA length, and $n$ as the truncated length.  Compare that to $n$ in 3.3.2 as the length of the non-truncated but overclocked RCA.}
      
      \Response In both sections, $k$ is used to denote the word-length of the input signal, whereas $n$ is the word-length of RCA.
      
  \item \Question{Also in 3.3.2, the MSB is given as $2^n$, instead of $2^0$ as would be expected for the [-1, 1) fixed-point format you mentioned.}
      
      \Response : $2^n$ is used here because $2^i$ is used in Equation (7). Nevertheless we agree it might be confusing to put $2^n$ here, and we have changed the second paragraph of Section 3.3.2 to look as follows:\\
      
      \PaperText{For $C_{tm}$, correct results will be generated from bit $S_t$ to bit $S_{t+b-1}$. Hence the absolute value of error seen at the output, normalized to the MSB, is given by (7), where $\hat{S}_i$ and $S_i$ denote the actual value and error-free value of outputs at bit $i$, respectively.}\\
      
  \item \Question{Still in 3.3.2, you seem to be saying that when a timing error occurs, none of the bits propagate. I think you are assuming the worst case, where no carry propagation occurs, so it is no surprise that you comment that the magnitude of overclocking error has no dependence on the length of carry chain $m$.}
      
      \Response To be added.
      
  \item \Question{In 3.3.4, I see the $E_O = 2^{-b}-2^{-n-1}$ result, but am puzzled by it.  At the very least, I am not sure where the $-b$ exponent is coming from, since (8) defines $e_{tm} = 2^{t+b-n}$.}
      
      \Response We would like to derive Equation (14) as follows:
      xxx
      
  \item \Question{The third and fourth paragraphs of Section 5.1 mention the 4-stage CSA, but Figure 4 does not show any data for a 4-stage CSA.}
      
      \Response Thanks. In the third paragraph of Section 3.1, it should be the 3-stage CSA. Therefore the last sentence of this paragraph is revised as:\\
      
      \PaperText{Although the CSA with 3 stages is best for some frequencies, the overclocked RCA is still the optimum design when high operating frequencies are applied.}\\
      
      The 4-stage CSA is introduced in the fourth paragraph and Fig. 5 and Fig. 6, where we consider a variety of area constraints. Ideally with larger available area, CSA with more stages will be included in both Fig. 5 and Fig. 6. However the general trend will be similar. In previous paragraphs of this section, we only use 2-stage CSA and 3-stage CSA as examples to illustrate the design method. We have clarified this point by adding the following sentence to the fourth paragraph of Section 3.1:\\
      
      \PaperText{We implement CSA with all possible stage numbers within the given area specification.}\\
      
  \item \Question{In the experimental setup, I would be curious to know whether you used any particular constraints on the circuits-under-test to enhance their chances of success.}
      
      \Response Actually no specific constraints are applied, except the global clock constraint.
      
  \item \Question{May I assume that your figures will be colorized?}
        
      \Response Yes, Fig.~3, Fig.~4 and Fig.~10 will be colorized.
      
  \item \Question{An incomplete list of spelling or grammar or punctuation comments.}
      
      \Response Thanks, they have been revised.

      

\end{enumerate}




\end{document}